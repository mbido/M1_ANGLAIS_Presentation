\documentclass{beamer}

\usetheme{Madrid}

\usepackage[utf8]{inputenc}
\usepackage[T1]{fontenc}
\usepackage{amsmath}
\usepackage{amsfonts}
\usepackage{amssymb}
\usepackage{tikz}
\usetikzlibrary{quantikz}
\usepackage{lineno}

\title{Quantum Computing and Cryptography}
\author{Damien, Théo, Matthieu}
\institute{University}
\date{February 12, 2025}

\begin{document}

\begin{frame}
\maketitle
\end{frame}

\begin{frame}{Outline}
\tableofcontents
\end{frame}

\section{Intro}
\begin{frame}{Introduction}
\begin{linenumbers}

\end{linenumbers}
\end{frame}

\section{RSA}
\begin{frame}{RSA}
\begin{linenumbers}

\end{linenumbers}
\end{frame}

\section{Quantique}

\subsection{Introduction au quantique}
\subsubsection{Qubits}
\begin{frame}{Qubits}
\begin{linenumbers}
    \begin{itemize}[<+->]
        \item Classical bit $\rightarrow$ 0 or 1
        \item Qubit $\rightarrow$ vector $\in \mathbb{C}^2$
        \item $\bra0 = \begin{bsmallmatrix} 1 & 0\end{bsmallmatrix}$
        \item $\bra1 = \begin{bsmallmatrix} 0 & 1\end{bsmallmatrix}$
        \item $\bra+ = \frac{1}{\sqrt{2}}\begin{bsmallmatrix} 1 & 1\end{bsmallmatrix}$
        \item $\bra- = \frac{1}{\sqrt{2}}\begin{bsmallmatrix} 1 & -1\end{bsmallmatrix}$
    \end{itemize}
\end{linenumbers}
\end{frame}

\subsubsection{Mesures}
\begin{frame}{Mesures}
\begin{linenumbers}
  \begin{itemize}[<+->]
    \item $\bra0$ => 0 (100\%)
    \item $\bra1$ => 1 (100\%)
    \item $\bra+$ => 0 (50\%), 1 (50\%)
    \item $\bra-$ => 0 (50\%), 1 (50\%)
  \end{itemize}
\end{linenumbers}
\end{frame}

\subsubsection{Gates}
\begin{frame}{Gates - Slide 1}
\begin{linenumbers}
    \begin{itemize}[<+->]
        \item Gate $X$
        \begin{itemize}
            \item $X\ket{0} \rightarrow \ket{1}$
            \item $X\ket{1} \rightarrow \ket{0}$
        \end{itemize}
        \item Circuit
        \begin{quantikz}
            \lstick{$\ket{0}$} & \gate{X} & \meter{}
        \end{quantikz}
    \end{itemize}
\end{linenumbers}
\end{frame}

\begin{frame}{Gates - Slide 2}
\begin{linenumbers}
    \begin{itemize}[<+->]
        \item Gate $H$
        \begin{itemize}
            \item $H\ket{0} \rightarrow \ket{+}$
            \item $H\ket{1} \rightarrow \ket{-}$
        \end{itemize}
        \item Circuit
        \begin{quantikz}
            \lstick{$\ket{0}$} & \gate{H} & \meter{}
        \end{quantikz}
    \end{itemize}
\end{linenumbers}
\end{frame}

\subsection{Algorithmes quantique}
\subsubsection{Problème B.V}
\begin{frame}{Problème B.V}
\begin{linenumbers}
Given the oracle of a function $f$ :

$f : \{0, 1\}^n \rightarrow \{0, 1\}$
$f(x) = x\cdot s$

Find $s$ in the few request possible.
\end{linenumbers}
\end{frame}

\subsubsection{Algo classique}
\begin{frame}{Algo classique - Slide 1}
\begin{linenumbers}
with $n=2$
try :
\begin{itemize}[<+->]
    \item $f(10) = s_0$
    \item $f(01) = s_1$
\end{itemize}

2 requests.
\end{linenumbers}
\end{frame}

\begin{frame}{Algo classique - Slide 2}
\begin{linenumbers}
in general :
$\mathcal{O}(n)$ $\rightarrow$ Try every $x$ that contains one bit to 1. At each query, we get the value of that bit in s
\end{linenumbers}
\end{frame}

\subsubsection{Algo Quantique}
\begin{frame}{Algo Quantique - Slide 1}
\begin{linenumbers}
$\mathcal{O}(1)$ $\rightarrow$ Just try every $x$ at the same time.

Not only the $x$ with only one bit at one but every possible $x$.
\end{linenumbers}
\end{frame}

\begin{frame}{Algo Quantique - Slide 2}
\begin{linenumbers}
\begin{quantikz}
    \lstick{$\ket{0}$} & \gate{H} & \qw & \qw & \gate{H} & \meter{} \\
    \lstick{$\ket{0}$} & \gate{H} & \ctrl{-1} & \gate{H} & \qw & \meter{}
\end{quantikz}
\end{linenumbers}
\end{frame}

\subsubsection{Shor}
\begin{frame}{Shor}
\begin{linenumbers}
    \begin{itemize}[<+->]
        \item Gain de complexité :
        $\mathcal O(e^b)$ $\rightarrow$ $\mathcal O(b³)$
        \item combien de qubit il faut
        \item combien de cubit on as
    \end{itemize}
\end{linenumbers}
\end{frame}

\section{Post-Quantique}
\begin{frame}{Post-quantique}
\begin{linenumbers}

\end{linenumbers}
\end{frame}

\section{Conclusion}
\begin{frame}{Conclusion}
\begin{linenumbers}
    % Content from brouillon.md - To be filled
\end{linenumbers}
\end{frame}

\end{document}
